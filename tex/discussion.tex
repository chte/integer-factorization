We chose to work with quadratic sieve since it's one of the most efficient integer factorization algorithms out there. It is the fastest algorithm for factorization up to 100 bit integers, the challenge was obvious but since we had knowledge that Pollard's Rho combined with Brent's cycle finding granted approximately 75 points in KATTIS and that the code was not a big challenge to write we wanted to give ourselves a challenge and dive in to a front-running algorithm.
The result was as expected from a correctly implemented quadratic sieve method, however, on the first submitt (that was working) we got a score of 85 points which was due to partly a lack of general optimization but the main reason was making sure the smoothness bounds worked correctly. There are a lot of mathematical parts to implementing this algorithm, altough not very complex math in theory getting all the parts of the algorithm to work well was a tough job. The understanding of the concept came as we went along and step-by-step we improved our code and realized the earlier errors we made and in the end realized that the concept is not hard to graps if you divide the problem in to subproblems and follow pre-existing formulas to execute it.