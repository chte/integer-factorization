{\bf Notation.} By 'log$_b$' we denote the base $b$ logarithm and the natural logarithm denotes by ln=log$_e$ with $e \approx 2.71828$. 
The largest integer $\leq x$ is denoted by '[x]'.  The number of primes $\leq x$ is denoted by '$\pi(x)$', and due the \emph{Prime number theorem}\cite{Hardy2008}
we know that $\pi(x) \approx x / $ln$(x)$ .

{\bf Smoothness.} A positive integer is \emph{B-smooth} if all its prime factors are lesser than a boundary $B$.
Furthermore an integer, say $x$, is said to be \emph{smooth with respect to S}, if $x$ can be completely factored using
integers by some set $S$ alone.


{\bf Modular arithmetic.} Throughout this paper '$x \equiv y$ mod $n$' means that $(x-y)$ is a multiple of $n$ whereas $x,y$ and $n \in \mathbb{N}_{\ne 0}$.

Similarly, '$x \not\equiv y$ mod $n$' mean that $(x-y)$ is not a multiple of $n$. 
Euclid's algorithm for finding the \emph{greatest common divisor} of two non-negative integers, say x and z, is denoted by gcd$(x,z)$.
