\documentclass[twoside,11pt,english]{article}


% ------
% Fonts and typesetting settings
\usepackage[sc]{mathpazo}
\usepackage[utf8]{inputenc}
\usepackage[T1]{fontenc}
\linespread{1.05} % Palatino needs more space between lines
\usepackage{microtype}
\usepackage{babel}
\usepackage{graphicx}
\usepackage{amsmath,amsthm,amssymb}
\usepackage{graphicx}
\usepackage[font=it]{caption}% http://ctan.org/pkg/caption
\usepackage{algorithmic}% http://ctan.org/pkg/{multicol,lipsum,graphicx,float}
\usepackage{booktabs,dcolumn} % Tables.

% ------
% Page layout
\usepackage[hmarginratio=1:1,top=32mm,columnsep=20pt]{geometry}
\usepackage[font=it]{caption}
\usepackage{paralist}
\usepackage{multicol}

% ------
% Abstract
\usepackage{abstract}
	\renewcommand{\abstractnamefont}{\normalfont\bfseries}
	\renewcommand{\abstracttextfont}{\normalfont\small\itshape}


% ------
% Titling (section/subsection)
\usepackage{titlesec}
\renewcommand\thesection{\Roman{section}}
\titleformat{\section}[block]{\large\scshape\centering}{\thesection.}{1em}{}


% ------
% Header/footer
\usepackage{fancyhdr}
	\pagestyle{fancy}
	\fancyhead{}
	\fancyfoot{}
	\fancyhead[C]{}
	\fancyfoot[RO,LE]{\thepage}


% ------
% Clickable URLs (optional)
\usepackage{hyperref}

% ------
% Maketitle metadata
\title{\vspace{-15mm}%
	\fontsize{24pt}{10pt}\selectfont
	\textbf{Factoring large integers using Quadratic Sieve}
	}	
\author{%
	\large
	\textsc{Christopher Teljstedt} \\[2mm]
	\normalsize	\href{mailto:chte@kth.se}{chte@kth.se} 
	\and
	\textsc{Henrik Viksten} \\[2mm]
	\normalsize	\href{mailto:hviksten@kth.se}{hviksten@kth.se}
	}

% ------
% Macros
\providecommand{\myceil}[1]{\left \lceil #1 \right \rceil }
\providecommand{\myfloor}[1]{\left \lfloor #1 \right \rfloor }

\makeatletter
\def\imod#1{\allowbreak\mkern10mu({\operator@font mod}\,\,#1)}
\makeatother

%%%%%%%%%%%%%%%%%%%%%%%%
\begin{document}

\maketitle
\thispagestyle{fancy}
\begin{abstract}

\noindent This report will give you an insight to factorization of integers using quadratic sieve. It covers the basics of the factoring problem and mentions other commonly used algorithms, but the main focus is on the quadratic sieve which is the best algorithm for this problem up to 110 bit integers. \\
This report was written during the course DD2440 Advanced Algorithms at KTH 2013 and recieved the maximum amount of points in the scoring system KATTIS.

\end{abstract}
\newpage

\tableofcontents
\newpage

\begin{multicols}{2}
\section{Introduction}
This report will cover the work of a integer factorization program written in the course DD2440 advanced algorithms. It was decided to focus primarily on quadratic sieve which should give a good understand of the problem in general.

\subsection{Purpose}
The goal of the program is to be able to factorize integers in an efficient way and also pass KATTIS test cases with as high score as possible, this is done by improving the program step-by-step and gradually increasing the performance and intelligence in the task at hand. In the report the methods used will be analyzed described. How they work independently and their correlation in our implementation.

\subsection{Problem}
An unknown set of 100 integers in varying size are used as input to the algorithm from KATTIS. The output is either the factors of the input if it is solved or the string fail otherwise. If number $n_i = p_1^{k_1} \ldots p_m^{k_m}$, then the program should print the following on \texttt{stdout}.
\begin{align*}
&\left.\begin{aligned}
      &p_1 \\
      &p_1 \\
      &\vdots \\
      &p_1 
      \end{aligned}
\right\}
\qquad k_1 \text{ times} \\
&\vdots \\
&\left.\begin{aligned}
      &p_m \\
      &p_m \\
      &\vdots \\
      &p_m 
      \end{aligned}
\right\}
\qquad k_m \text{ times}
\end{align*}

One of the problems is dealing with large integers within the timeframe but also finding every single non-trivial factor. The restrictions in kattis are 15 seconds and 64MB of memory.
\subsection{Scope}
There is no intention of reinventing any methods that already exists, the idea is to create a solver that can factorize unknown integers and in the end get a high score on KATTIS.

\subsection{Statement of Collaboration}
The project was done as a group, although since Christopher is more knowledgable in the C++ syntax he took care of most of the physical coding with Henrik by his side. The learning, investigating and writing of the report was done as a duo.
\section{Preliminaries}
{\bf Notation.} By 'log$_b$' we denote the base $b$ logarithm and the natural logarithm denotes by ln=log$_e$ with $e \approx 2.71828$. 
The largest integer $\leq x$ is denoted by '[x]'.  The number of primes $\leq x$ is denoted by '$\pi(x)$', and due the \emph{Prime number theorem}\cite{Hardy2008}
we know that $\pi(x) \approx x / $ln$(x)$ .

{\bf Modular arithmetic.} Throughout this paper '$x \equiv y$ mod $n$' means that $(x-y)$ is a multiple of $n$ whereas $x,y$ and $n \in \mathbb{N}_{\ne 0}$.
Similarly, '$x \not\equiv y$ mod $n$' mean that $(x-y)$ is not a multiple of $n$. 
Euclid's algorithm for finding the \emph{greatest common divisor} of two non-negative integers, say x and z, is denoted by gcd$(x,z)$.

\section{Background}
The Integer Factorization problem is defined as follows; given a composite integer $n$, find any non-trivial factor $e$ of $n$, such that $e|n$. At first it might seem like a trivial task, but for large integers this has proven to be a very difficult task.

\subsection{Trial Division}
 The most straight-forward approach to factoring composite numbers is \emph{trail division}, which essentially (just like its name suggests) just check for every prime number $p \leq \sqrt{n}$ if $p|n$. 

Improvement is to only do tests when $p$ is a prime number, i.e. keep a list of precomputed prime numbers. Although trial division is easy to implement and guaranteed to grant an answer it is not efficient with large integers. There is improvements that can be applied such as storing a precomputed tables of primes to bring up the speed. However, this instead requires alot of memory during run-time as well as storage, when dealing with larger integers. A

\subsection{Pollard's $\rho$}
In 1975 John M. Pollard proposed a new and very efficient Monte Carlo algorithm for factoring integers, now known as Pollard's $\rho$ (rho) method.
It was a break through and proved to be alot more faster than its predecessor trial division for finding small non-trivial factors of a large integer $n$.

\subsubsection{Basic idea}
Pollard's $p$ basic concept is that a sequence of pseudo-random integers constructed as

\begin{equation}
x_0 = \texttt{rand}(0,\ n-1) 
\end{equation}

\begin{equation}
x_i = f(x_{i-1}) \ (\texttt{mod} \ n), \ \texttt{for} \ i = 1, 2, \ldots
\end{equation}

where $f$ is polynomial which in most practical implementations has the form $f(x) = x^2 + \alpha, \alpha \neq 0, -2$.
The key observation that if we consider a non-trivial divisor of $n$, say $d$, that is small compared to $n$. Then there exists smaller congruence
groups modulo $d$ compared to $n$. Because of this there is a probability that there exists two $x_i$ and $x_j$ such that $x_i \equiv x_j$ (mod $d$),
while $x_i \not\equiv x_j$ (mod $n$). Thus it follows from that gcd$(x_i - x_j, n)$ is a non-trivial factor of $n$.

The idea of pollard rho algorithm is to iterate a formula until it falls into a cycle. We want to find a $x$ and $y$ whereas the $x$ makes twice as many iterations as the $y$ using a function (mod $n$) as a generator of a pseudo-random sequence. The $gcd(x - y, n)$ is taken each step, and it reaches $n$ we have not found an answer and the algorithm terminates with a failure.

We can write $n = p \cdot q$, when $x$, which iterates twice as fast as $y$, catches up with $y$ which will happen eventually, at this point factor $p$ will be found. The time it will take cannot be proven matematically and can only be proven by heuristics. If the sequence behaves randomly it would take approximately p steps to find p, which is not very effcient. \cite{avalg} 

\includegraphics[scale = 0.5]{pollards.png}

The figure above illustrates the Pollard's Rho cycle. The mapping of $x_{i+1}$ is instead replaced with a function $x^2+1$ so that we get $x^2_{i}+1$  $mod$ $N$ the factor $p$ will be found after $O(\sqrt{p})$  $\epsilon$ $O(N^{1/4})$ steps. \cite{avalg}\\

Pollard's rho algorithm can be improved further by implementing Brent's cycle finding method. \cite{brent}

\subsection{Miller-Rabin primality test}
Miller-Rabin primality test is a probabalistics algorithm which determines if the given input $N$ is a prime. It is based on the properties of strong pseudoprimes, given an integer $N$, $N = 2^r * s + 1$ where $s$ is odd, you choose a random number $a$ with the properties $1 \leq a \leq N - 1$. When $a^s \equiv 1$ $mod$ $N$ or $a^{2js} \equiv - 1$ $mod$ $N$ where $0 \leq j \leq r - 1$, if the input number $N$ is a prime it will pass the test with any random number $a$.

<<<<<<< HEAD
Since Miller-Rabin is a probabalistics method it is not completely true that $N$ is a prime simply by passing the test, however the probability that the answer is true when $N$ is a composite number is $1 / 4^{N}$ which grows quickly with $N$. It can be considered a very small trade-off to use a probabalistics method because the algorithm executes at $O(k$ $log^3 N)$ where $k$ is the number of different values of $a$ tested. \cite{miller}
=======
Since Miller-Rabin is a probabalistics method it is not completely true that $N$ is a prime simply by passing the test, however the probability that the answer is true when $N$ is a composite number is $1 / 4^{N}$ which grows quickly with $N$. It can be considered a very small trade-off to use a probabalistics method because the algorithm executes at $O(k$ $log^3 N)$ where $k$ is the number of different values of $a$ tested. \cite{miller}
>>>>>>> 74c414c56f71a8c92c08abd685ce1136c9b650a6

\section{Quadratic Sieve}
\subsection{Brief explaination}
The quadratic sieve algorithm is today the algorithm of choice when factoring very large composite numbers
with no small factors. The general idea behind quadratic factoring is based on Fermat's observation that
a composite number $n$ can be factored if one can find two find two integers $x,y \in Z$, such
that $x^2 \equiv x^2$ (mod $n$) and $x \not\equiv \pm y$ (mod $n$). This would imply that,
\begin{equation}
n \ | \ x^2-y^2 = (x-y) \cdot (x+y)
\end{equation}
but $n$ neither divides $(x-y)$ nor $(x+y)$. 
Furthermore it can be rewritten as $(x-y) \cdot (x+y) = k \cdot p \cdot q$ for some integer $k$, thus becoming two possible cases.

\begin{itemize}
	\item either $p$ divides $(x-y)$ and $q$ divides $(x+y)$, or vice versa.
	\item or both $p$ and $q$ divides $(x-y)$ and neither of them divides $(x+y)$, or vice versa.
\end{itemize}

Hence, the greatest common divisor of ($x-y,n$) and ($x+y,n$) would in first case yield $p$ or $q$ and a non-trivial factor of 
$n$ is found. In the second case we get $n$ or $1$ and trivial solution is found. There is atleast $1/2$ probabilty of 
the solution being non-trivial\cite{Pomerance1985}.

Carl Pomerance suggested a method to find such squares\cite{Pomerance1985}. The first step in doing so is to is to define the polynomial 
\begin{equation}
q(r) = (r + \myfloor{\sqrt{n}})^2-n \approx \tilde{r}^2-n
\end{equation}
Now, consider we set of primes $P = \{ p_1, p_2, ..., p_k \}$ lesser than a bound $B$, i.e $k < \pi(B)$. We then want to construct a subset of integers  $r_1, r_2, \ldots, r_k$ such that $\forall i : q(r_i) = r_i^2-n$ (mod $n$) is smooth in respect to $P$, more specifically $\forall i$ :

\begin{equation}
 q(r_i) = r_i^2 \equiv p_1^{e_{i1}} \cdot \ldots \cdot p_k^{e_{i\pi(B)}} \ (\texttt{mod} \ n)
\end{equation}

where $e_{ij}$ is the exponent of $p_j$ of factorization of $q(r_i)$. If the exponents for all primes sums to
an even number we arrove at following relation:

\begin{equation}
 \prod_{i=1}^{n} q(r_i) = \prod_{i=1}^{n} r_i^2 \equiv (p_1^{e_{1}} \cdot \ldots \cdot p_k^{\pi(B)})^2 \ (\texttt{mod} \ n)
\end{equation}

and the integers we $x$, $y$ we sought to find are simply:

\begin{equation}
 x = \prod_{i=1}^{n} q(r_i)
\end{equation}

\begin{equation}
y = \prod_{i=1}^{\pi(B)} p_1^{e_{i}}
\end{equation}

Lets consider a factor base $P$ with the primes $p_1, p_2, ..., p_k$ that is $k \geq B$ and coprime to $n$.
We want to search for small $r_i$ so that $q(r_i)$ is smooth in respect to $P$. If we find such $r$'s we say that $q(r_i) = (r + \myfloor{\sqrt{n}})^2-n$ is B-smooth and we can factor completely over the factor base.

A prime factorization $p_1^{e_1} \cdot p_2^{e_2} \cdot \ldots \cdot p_k^{e_{\pi(B)}}$ of a B-smooth number can then be expressed be entirely expressed by the factor base and exponent vector $(e_1, e_2, \ldots, e_{\pi(B)})$.

Since a prime $p$ only divides $q(r_i)$ if and only if it divides $q(r_i+kp)$ for any integer
$k$, we can found these values efficiently using a sieve. For this reason its called the $sieving step$ and because only primes $(\frac{n}{p_i}) = 1$ can divide $q(r_i)$, explains the definition of the factor base $P$. 
Naivly one could simply randomly select $r$ in a range of interest and verify that $q(r)$ is divisable by all the primes in the factor base. Instead in quadratic sieve algorithm we can first solve the quadratic congruence $r^2 \equiv n$ (mod $p$) and then cleverly only divide by prime numbers corresponding to the interval $r$ and $r+kp$ for some any integer $k$. Thus, finding these $r$'s becomes easier than just a simple trail division as we apriori know at that a B-smooth $q(r_i)$ is divisable by a subset of primes in $P$. This is the reason why trail division of B-smooth numbers preferable over is a plain naive trail division.

Then the product of subsequence of $x$ i.e. $q(x_1) \cdot q(x_2) \cdot \ldots \cdot q(x_k)$ produced a 
square $iff$ the exponent vectors has only even entries. That is that it's the \emph{null}-vector in mod 2
space. So with the collection of smooth numbers we want to form their exponent vectors, reduce them modulo 2. For that vector space (finite field of two elements) the sequence of vectors thus becomes linearly dependent and can be solved with Gaussian Eliminatio to find the non-empty subsequences. The exponent vectors is represented as the rows of a matrix. Hence, the column sums must be even.

\subsection{Deciding on factor base}
To factorize during the sieving step we want all the $q(r)$'s to be smooth in respect
to the factore base, i.e. $q(r)$ must be divisable by only primes in the factor base.
If a prime $p$ divides $q(r_i)$ earlier mentioned polynomial implies following

\begin{equation}
(r_i + \myfloor{\sqrt{n}})^2 \equiv n \ \texttt{mod} \ p_i
\end{equation}

Hence, $n$ is a quadratic residue modulo $p_i$ and we only need to consider those primes. Thus the set of primes $p_i$ for which the Legendre symbol $(\frac{n}{p_i})$ is 1 will form the factor base\cite{Pomerance1985}.


\subsection{Sieving step}
Only a small fraction of numbers will be completely factorized by the primes of
the factor base. Therefore, a crucial step is to sieve as many numbers as possible.
From following observation we can decide on intervals of primes in $P$ that
to divide prime numbers of $q(r)$.

\begin{align}
q(r) &= \tilde{r}^2 - n, \ \tilde{r} = (\myfloor{\sqrt{n}} + r) \\
q(r+kp) &= (\tilde{r}+kp)^2-n \\
q(r+kp) &= \tilde{r}^2 + 2 \ \tilde{r}kp + (kp)^2 - n \\
q(r+kp) &= q(r) + 2\tilde{r}kp + (kp)^2 \\
		&\equiv q(r) \imod{p}
\end{align}

So the solution $q(r) \equiv 0 {p}$ for $r_i$ yields a sequence $q(r_i)$ 
that is divisable by $p$. This can be solved using Shanks-Tonelli
algorithm\cite{Pomerance1985}. Thus we obtain two solutions which we call $idx_{1p}$
and $idx_{2p} = p - idx_{1p}$. Then those $q(r_i)$ with $r_i$'s
in the sieving interval are divisable by $p$ when $r_i = idx_{1p}, idx_{2p}+pk$
for some integer $k$.

\paragraph{Gauss-eliminatio}
If $q(r_i)$ does completely factor, the next step is to
test different linear combination of these so that it yields perfect square for 
product of $q(r_i), i \in [1,k]$. In other words we want to find solutions to

\begin{equation}
q(r_1)e_1 + q(r_2)e_2 + \ldots + q(r_k)e_k
\end{equation}

where $e_i$ is either 0 or 1. This means that if $\vec{a}_i$ is the
row in a matrix $A$ corresponding to $q(r_i)$ then we get

\begin{equation}
\vec{a}_1 e_1 + \vec{a}_2 e_2 + \ldots + \vec{a}_k e_k
\end{equation}

and we have to solve 

\begin{equation}
\vec{e}_1 A^T = \vec{0} \ (\texttt{mod} N)
\end{equation}

where $\vec{e}$ is the exponent vector. Then the task simply becomes to find the 
null-space of $A^T$, with Gauss Elimination in the field $GF[2]$ so that all calculations 
can be made modulo $2$.

To ensure linear dependancy there have to be more B-smooth numbers $r$ than the number of primes 
in the factor base. We want to produce linear combination so that each prime power in the combination is even, 
only then we have a perfect square. 

From this stage what remains is to try combinations until we have
a solution of linear combinations yielding a product that is perfect square,
with equation (7) and (8). To ensure that $x$ and $y$ are not trivial solutions
we can make use of equation (1) and check the requirement $x \not\equiv \pm y$ (mod $n$),
since there is otherwise a $1/2$ probability of gcd($x-y,n$) resulting in a 
trivial solution. If a trivial solution is found we just rerun and try another linear
combination until we find a non-trivial one.



\section{Implementation}
For our implementation we used several papers as reference. One few that really come handy
lecture notes by Carl Pomerance \cite{Pomerance05smoothnumbers}
\section{Results}
These tests were solely based on the score and running time on the 
\texttt{KATTIS} problem \texttt{oldkattis:factoring}.

\paragraph{Primality testing}
At first we just tried to do a primality test with Miller-Rabin. 
The results can be seen in table

 \textbf{Score} 1   \textbf{Running time} 0.01s 
                          

\paragraph{Quadratic sieve}

We implemented a naive Quadratic Sieve from start and
at first with bad B smooth boundary. It had
unoptmizied sieving functions, we did handle perfect powers.

This yielded a result on Kattis as shown below.

 \textbf{Score} 85   \textbf{Running time} 13.56s 
                          

By improving on the problems mentioned above and tweaking the smoothness boundary
B which regulates the factor base size aswell as the number of linear relations,
i.e how many extra smooth numbers, we managed to get 100 points score on Kattis.

 \textbf{Score} 100   \textbf{Running time} 4.32s 

\subsection{Kattis submission}
The best Kattis submission has the ID:459013.
\section{Discussion}
We chose to work with quadratic sieve since it's one of the most efficient integer factorization algorithms out there. It is the fastest algorithm for factorization up to 100 bit integers, the challenge was obvious but since we had knowledge that Pollard's Rho combined with Brent's cycle finding granted approximately 75 points in KATTIS and that the code was not a big challenge to write we wanted to give ourselves a challenge and dive in to a front-running algorithm.
The result was as expected from a correctly implemented quadratic sieve method, however, on the first submitt (that was working) we got a score of 85 points which was due to partly a lack of general optimization but the main reason was making sure the smoothness bounds worked correctly. There are a lot of mathematical parts to implementing this algorithm, altough not very complex math in theory getting all the parts of the algorithm to work well was a tough job. The understanding of the concept came as we went along and step-by-step we improved our code and realized the earlier errors we made and in the end realized that the concept is not hard to graps if you divide the problem in to subproblems and follow pre-existing formulas to execute it.
\section{Conclusion}
In hindsight there is no regret of choosing the more complex solution to the problem, it did not only give us more understanding of how to solve the problem we were given in an intelligent way but we also managed to successfully implement the quadratic sieve algorithm to achieve maximum score in KATTIS. Of course the solution can be optimized further but considering the time constraints we are satisfied with our work.

\newpage
\bibliographystyle{plain}
\bibliography{References}

\end{multicols}

\end{document}
