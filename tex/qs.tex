\subsection{Brief explaination}
The quadratic sieve algorithm is today the algorithm of choice when factoring very large composite numbers
with no small factors. The general idea behind quadratic factoring is based on Fermat's observation that
a composite number if one can find two find two integers $x,y \in Z$, such
that $x^2 \equiv y^2$ (mod $n$) and $x \not\equiv \pm y$ (mod $n$). This would imply that,
\begin{equation}
n \ | \ x^2-y^2 = (x-y) \cdot (x+y)
\end{equation}
but $n$ neither divides $(x-y)$ nor $(x+y)$. 
Furthermore it can be rewritten as $(x-y) \cdot (x+y) = k \cdot p \cdot q$ for some integer $k$, thus becoming two possible cases.

\begin{itemize}
	\item either $p$ divides $(x-y)$ and $q$ divides $(x+y)$, or vice versa.
	\item or both $p$ and $q$ divides $(x-y)$ and neither of them divides $(x+y)$, or vice versa.
\end{itemize}

Hence, the greatest common divisor of ($x-y,n$) and ($x+y,n$) would with with a 1/2 probability yield first case which is $p$ or $q$ and a non-trivial factor of 
$n$ is found. In the second case we get $n$ or $1$ and trivial solution is found. 

Carl Pomerance suggested a method to find these tsuch squares\cite{Pomerance1985}. The first step in
doing so is to is to define the polynomial 
\begin{equation}
Q(x) = (x + \myfloor{\sqrt{s}})^2-N 
\end{equation}