\subsection{Brief explaination}
The quadratic sieve algorithm is today the algorithm of choice when factoring very large composite numbers
with no small factors. The general idea behind quadratic factoring is to find two integers $x,y \in Z$, such
that $x^2 \equiv y^2$ and $x \not\equiv \pm y$ mod $N$. This would imply that $x^2-y^2 = (x-y) \cdot (x+y) \equiv 0$ mod $N$ and 
that $(x-y) \cdot (x+y) = k \cdot p \cdot q$ for some integer $k$. Furthermore this can be seperated into to two possible cases.

\begin{itemize}
	\item either $p$ divides $(x-y)$ and $q$ divides $(x+y)$, or vice versa.
	\item or both $p$ and $q$ divides $(x-y)$ and neither of them divides $(x+y)$, or vice versa.
\end{itemize}


Calculating $d=$gcd($x-y,N$) would then in first case yield $d = p$ or $d = q$ and a non-trivial factor of 
$N$ is found. In the second case we get $d = N$ or $d = 1$ and trivial solution is found. In conclusion,
there a 1/2 probability to factor $N$.

Carl Pomerance suggested a method to find these two distinct integers\cite{Pomerance1985}. The first step in
doing so is to is to define the polynomial 
\begin{equation}
Q(x) = (x + \myfloor{\sqrt{s}})^2-N 
\end{equation}