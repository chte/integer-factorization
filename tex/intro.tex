This report will cover the work of a integer factorization program written in the course DD2440 advanced algorithms. Methods that can be used to solve the integer factorization problem in a effective way using existing algorithms and an explanation of methods used.

\subsection{Purpose}
The goal of the program is to pass kattis test cases with as high score as possible, this is done by improving the program step-by-step and gradually increasing the performance. In the report the methods used will be analyzed. How they work independently and their correlation in our implementation.

\subsection{Problem}
An unknown set of 100 integers in varying size are used as input to the algorithm from kattis. The output is either the factors of the input in case it is solved or the string fail otherwise. One of the problems is dealing with large integers within the timeframe but also finding every single non-trivial factor. The restrictions in kattis are 15 seconds and 64MB of memory.

\subsection{Scope}
There is no intention of reinventing any methods that already exists, the idea is to create a solver that can factorize unknown integers and in the end get a high score on kattis.

\subsection{Statement of Collaboration}
Some text.